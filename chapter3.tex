% Chapter 2

\chapter{LITERATURE SURVEY} % Write in your own chapter title


Predicting code characteristics or extracting relevant aspects from large volumes of code data has progressed significantly in recent years. The act of predicting code properties without compiling or running is used for name prediction of program entities \cite{I}, code generation \cite{J}, code completion \cite{K} and code summarization \cite{L}. Representing the code suitable for a learning system to understand has been performed using two ways, either by using Code Embeddings representation or using the AST (Abstract Syntax Tree) representation. 

The goal of research in embeddings-based algorithms is to learn good code representations, compare source codes, and recommend ways to students. David Azcona et al. (2019) \cite{A} used program embeddings to profile students based on their code submissions. Their work compared the performances of different source code vectorization techniques to predict the correctness of a code submission. For research in AST representation, Mou et al. \cite{M} proposed a method for developing program vector using AST representation for use with Deep Learning models to classify computer programs. Other granularity levels for representations such as characters, tokens, and statements were investigated by the authors. 

In the field of providing feedback on code solutions, Piech et al. \cite{N} used Code Embeddings to provide feedback to students in Massive Online Open Courses (MOOC)s. They first learnt how to capture the functional and stylistic parts of student submissions, and then successively providing automatic feedback. This was accomplished by creating functionality matrices at each node in the submission's syntax tree. Paaßen et al. \cite{O} showed that a continuous hint strategy can predict what skilled students will do in a multi-step programming job, and that the hints created using embeddings can equal the edit hints offered by human instructors. Gross et al. \cite{P} used structured solution spaces to propose feedback strategies and automatic example assignments. Mou et al. \cite{Q} introduced a tree-based Convolutional Neural Network (TBCNN) that captures structural information using a convolution kernel constructed over program ASTs. They also utilised this method to classify programmes based on functionality and to detect code snippets that followed specific patterns. Furthermore, Proksch et al. \cite{R} demonstrated for C\# using solutions from GitHub, constructing a dataset of syntax trees can be used for suggestions.

J. Walker Orr et. al \cite{T} generated personalised feedback to students to improve the design of their program. Their focus was on design quality of programs. They annotated the student submissions with a design score between 0 and 1 using PyLint and considered the programs with design score over 0.75 as good programs. Around 40 features were extracted from the code's AST representation. Feedback was generated by comparing the feature vector of a code in question with the average feature vector of good programs.

In our work we propose a feedback generation system that leverages the use of AST representation along with a code's functionality and design to provide constructive feedback to student code submissions. 


 




