% Chapter 2

\chapter{LITERATURE SURVEY} % Write in your own chapter title

Predicting code characteristics or extracting relevant
aspects from large volumes of code data has progressed
significantly in recent years. Name
prediction of program entities \cite{I}, code generation
\cite{J}, code completion \cite{K} and code summarization
\cite{L} depend on anticipating and determining program's attributes and features without program compilation or program execution. Representing the code suitable for a learning
system to understand has been performed in two ways, either
using Code Embeddings representation or using the AST
(Abstract Syntax Tree) representation.

The goal of research in embeddings-based algorithms is to
learn good representations of programs, comparison of source programs, and
recommendation of ways for students to learn. David Azcona et al (2019)
\cite{A} tried to profile and cluster students based on their program submissions. Their work involved the comparison of many vectorization techniques of source programs to determine
the correctness of a program submitted by a student. For research in AST
representation, Mou et al \cite{M} proposed a method for
developing program vector using AST representation for use
with Deep Learning models to classify computer
programs. 

In the field of providing feedback on code solutions, Piech
et al \cite{N} used Embeddings of programs to provide feedback to learners in Massive Online Open Courses (MOOC). They first
learnt how to capture the functional and stylistic parts of
program submissions of learners, and successively provided
automatic feedback. This was accomplished by creating
functionality matrices at each node in the submission's
syntax tree. Paaben et al. \cite{O} demonstrated that a continuous hint approach can anticipate what expert students would perform in a programming assignment of multiple steps, and that the edit cues provided by human instructors can be equaled by embedding-based clues.
To offer feedback mechanisms and automatic example assignments, Gross et al \cite{P} used structured solution
spaces. Mou et al \cite{Q} introduced a tree-based
Convolutional Neural Network (TBCNN) that captures structural
information using a convolution kernel constructed over
program ASTs. This method was also used to classify programs based on their functionality and to detect code snippets that matched specified patterns. Furthermore, Proksch et al
\cite{R} showed for C\# programs by constructing a collection of syntax trees which were utilised for suggestions using solutions from GitHub.

To enhance the design of student programs, J Walker Orr et al \cite{T} designed a system that provided individualised feedback for pupils. Their
focus was only on the design quality of programs. They annotated the
student submissions with a design score between 0 and 1 and considered the programs with design score over
0.75 as good programs. Around 40 features were extracted from
the code's AST representation. Feedback was generated by
comparing the feature vector of a code in question with the
average feature vector of good programs.

In our work, we propose a feedback generation system that
leverages the use of AST representation along with a code's
functionality and design to provide constructive feedback to
student code submissions.


 




