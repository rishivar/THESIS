
\chapter{FEEDBACK GENERATION}

After prioritizing the best regression models for each problem, we
move on to using the selected model for generating feedback. The
feedback generation process follows a cyclic information flow where
the model is verified each time for a change depending on the input
change and the corresponding change is reflected as feedback. The
following subsections explain the process of how constructive feedback
is generated.

\section{GOLDEN FEATURE VECTOR}

The golden feature vector $F$ is the best feature vector that is
calculated for each problem. In retrospect, the golden feature vector
is the average of feature vectors of all code solutions submitted for
the program that have score greater than or equal to an empirically
decided threshold.

At any instance, the golden feature vector represents the correct and
accurate code solutions for a particular problem. The goal of this
vector is to act as a comparison tool for feature vectors that require
feedback/improvement.

\section{ALGORITHM}

Now, to generate feedback for a particular target program $x$ with
$x_{i}$ features where $i = 0,1,2, \ldots n$ ($n$ is the number of
features), we perform an iterative process of replacing one of the
features $x_{i}$ with the corresponding feature value $X_{i}$ from the
program's golden feature vector $F$. Now, the new modified feature
vector is passed as input into the regression model.

\begin{itemize}
\item If the output score improves after replacement, now we compare
  $x_{i}$ and $X_{i}$.
  \begin{itemize}
  \item If $x_{i} > X_{i}$, feedback is to decrease that particular
    feature in the program is advised.
  \item If $x_{i}<X_{i}$, feedback is to increase the particular
    feature in the program is advised.
  \end{itemize}
\item If the output score decreases after replacement, we consider the
  next feature in $x_{i}$
\item The above process is iterated until all the n features have been
  considered.
\end{itemize}

In each iteration, the goal of the algorithm is to make the target
feature vector closer to the golden feature vector. Thus, the changes
that are suggested as feedback are inferred to be improvements
for the target vector and constructive feedback is provided.
