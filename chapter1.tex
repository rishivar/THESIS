% Chapter 1

\chapter{INTRODUCTION} % Write in your own chapter title
%\label{fig:INTRODUCTION}
%\lhead{CHAPTER 1. \emph{INTRODUCTION}} % Write in your own chapter title to set the page header
Static Code Analysis (also known as Source Code Analysis) is usually performed as part of a Code Review (also known as white-box testing) Static Code Analysis in general refers to the use of Static Code Analysis tools which try to highlight potential vulnerabilities within ‘static’ (non-running) source code. This process provides an better understanding of structure of the code and can help ensure that the code adheres to industry standards. 

Software development teams and quality assurance teams employ static analysis to discover possible vulnerabilities.  While validating the code, the software will search all of the code in a project for vulnerabilities. Static analysis is often effective in detecting coding flaws such as programming errors, coding standards violations, and security flaws.

Static code analysis has several advantages, including improving code quality by evaluating all of the code in an application. When compared to manual code review, it allows for faster use of automated tools. Static testing, when combined with traditional testing methods, allows for greater debugging depth. Human error is less likely with automated tools. It will improve online or application security by raising the possibility of identifying vulnerabilities in the code.

We propose to use Machine Learning to perform Static Code Analysis (SCA) on student coding assignments to evaluate the code and give valuable feedback to improve the quality of code. We propose to accomplish this by training a new pipeline on our manually annotated student assignment data set. In our study, the pipeline will produce a design value score ranging from 1 to 10, which will be utilised to give personalised feedback explaining the logic behind the predicted design value score.

As a result, because feedback is a crucial part of effective learning, the review or feedback provided helps students improve the quality of their code. According to research, in the context of student assignments, feedback is more strongly and consistently connected with achievement than any other teaching practice. This link exists regardless of grade, financial status, race, or educational setting. Students' self-esteem, self-awareness, and drive to learn can all benefit from feedback. This method of providing feedback to students has been proposed as a strategy to improve learning and evaluation performance.