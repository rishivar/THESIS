% Chapter 1

\chapter{INTRODUCTION} % Write in your own chapter title
%\label{fig:INTRODUCTION}
%\lhead{CHAPTER 1. \emph{INTRODUCTION}} % Write in your own chapter title to set the page header
Static Code Analysis (also known as Source Code Analysis) is usually performed as part of a Code Review (also known as white-box testing) Static Code Analysis commonly refers to the running of Static Code Analysis tools that attempt to highlight possible vulnerabilities within ‘static’ (non-running) source code. The process provides an understanding of the code structure and can help ensure that the code adheres to industry standards. 

Static analysis is used in software engineering by software development and quality assurance teams. Automated tools can assist programmers and developers in carrying out static analysis. The software will scan all code in a project to check for vulnerabilities while validating the code. Static analysis is generally good at finding coding issues such as: Programming errors, Coding standard violations, security vulnerabilities. 

Some benefits of using static code analysis is increasing code quality as it can evaluate all the code in an application. It provides speed in using automated tools compared to manual code review. Paired with normal testing methods, static testing allows for more depth into debugging code. Automated tools are less prone to human error. It will increase the likelihood of finding vulnerabilities in the code, increasing web or application security. 

We propose to use Machine Learning to perform Static Code Analysis (SCA) on student coding assignments to evaluate the code and give valuable feedback to improve the quality of code. We propose to do this using a novel pipeline that is trained on our manually annotated  student assignment data-set. The pipeline will provide a design value score between 1 and 10 in our research which will be used to generate  individualized feedback explaining the reasoning behind the estimated design value score.


As a result, the review or feedback provided helps students improve the quality of their code as feedback is an important aspect of effective learning. Feedback is more strongly and consistently associated with accomplishment than any other teaching behavior in the context of student assignments, according to research. Regardless of grade, socioeconomic background, color, or school setting, this association exists. Feedback can boost a student's self-esteem, self-awareness, and desire to study. Giving students feedback in this way has been suggested as a way to improve learning and evaluation performance.
