% Chapter 3

\chapter{BACKGROUND AND MOTIVATION} % Write in your own chapter title
%

Feedback is an essential component of scaffolding for learning. Feedback provides insights into the assistance of learners in terms of achieving learning goals and improving self-regulated skills. The results highlight an increased engagement while performing the assessment, the usefulness of the feedback, as well as where the explanation was clear and where improvements are needed. 

Especially with the recent compulsion for online learning due the pandemic, feedback becomes even more critical since instructors and students are separated geographically and physically. In this context, feedback allows the instructor to customize learning content according to the students’ needs. However, giving feedback is a challenging task for instructors, especially in contexts of large cohorts. As a result, several automatic feedback systems have been proposed to reduce the workload on the part of the instructor. Although these systems have started gaining research attention.

The manual grading of assignments is a tedious and error-prone task, and the problem particularly aggravates when such an assessment involves a large number of students. The use of artificial intelligence can be useful to address these issues by automating the grading process, we can assist teachers in the correction and enable students to receive immediate feedback, thus improving their solutions before the final submission.

Unlike in-person education methods, online education tends to be more affordable. There’s also often a wide range of payment options that let you pay in installments or per class. This allows for financially disadvantaged students to upskill themselves to open avenues for better career opportunities. Money can also be saved from the commute and class materials, which are often available for free. In other words, the monetary investment is less, but the results can be comparable to other options. 

Online education enables the student to set their own learning pace, and there’s the added flexibility of setting a schedule that fits everyone’s agenda. As a result, using an online educational platform allows for a better balance of work and studies, so there’s no need to give anything up. Studying online makes finding a good work-study balance easier.

Students of online courses that lack a real-time tutor available can also reap the benefits of feedback with an assisting machine learning based Static Code Analyser that gives personalized feedback to students 


