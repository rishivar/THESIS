% Chapter 3

\chapter{BACKGROUND AND MOTIVATION} % Write in your own chapter title
%

\section{INTRODUCTION} % Write in your own chapter title
%\label{fig:INTRODUCTION}
%\lhead{CHAPTER 1. \emph{INTRODUCTION}} % Write in your own chapter title to set the page header
Code Review frequently involves a Static Code Analysis (SCA), often known as Source Code Analysis. In general, static code analysis refers to the use of SCA tools to identify possible vulnerabilities in 'static' (non-running) source code. This allows to get a clear picture of the code's structure and can aid in ensuring that the code meets industry standards.

Static code analysis is used by software development and quality assurance teams to find potential vulnerabilities. The SCA tool will scan all of the code in a project for vulnerabilities while verifying it. Static code analysis is frequently successful in finding coding problems such programming mistakes, coding standards breaches, and security issues.

Static code analysis has several advantages including
improving code quality by evaluating all of the code in an
application. In comparison to manual code review, automated tools consume very little time. When static testing is paired with typical testing methods, more debugging depth is possible. With automated technologies, human mistake is less likely. It will increase online or application security by increasing the likelihood of detecting code flaws.

We propose to use Machine Learning to perform Static Code
Analysis (SCA) on student coding assignments to evaluate the
code and give valuable feedback to improve the quality of
code. We propose to accomplish this by training a new
pipeline on our manually annotated student assignment data
set. In our study, the pipeline will produce a design value
score ranging from 1 to 10, which will be utilised to give
personalised feedback explaining the logic behind the
predicted design value score.

As a result, because feedback is a crucial part of effective
learning, the review or feedback provided helps students
improve the quality of their code. According to research, in
the context of student assignments, feedback is more strongly
and consistently connected with achievement than any other
teaching practice. This link exists regardless of grade,
financial status, race, or educational setting. Students'
self-esteem, self-awareness, and drive to learn can all
benefit from feedback. This method of providing feedback to
students has been proposed as a strategy to improve learning
and evaluation performance.

\section{MOTIVATION} % Write in your own chapter title
Feedback is an essential component of scaffolding for learning as
feedback assists students in achieving their learning goals and to
improve their self-regulation capability. The findings show that
participants were more engaged during the assessment when the feedback
was valuable and the explanations were clear and helpful.

Especially in recent situtaions of online learning due the pandemic,
feedback becomes even more critical since instructors and students are
separated geographically. In this case, feedback enables the
instructor to tailor learning content to the needs of the
pupils. Giving feedback, on the other hand, is a difficult chore for
instructors, especially in large groups. As a result, numerous
automatic feedback systems have been proposed in order to lessen the
instructor's labor.

Manually grading assignments is a time-consuming and error-prone process. Since grading is typically done for a large number of
student submissions, there is a increased probability of errors. Artificial intelligence (AI) techniques can help to solve these problems by automating the grading process. Teachers can be assisted with corrections and can provide instant feedback to students, allowing them to enhance their programming assignment solutions before submitting their final submission.

\iffalse
Online education in general is less expensive and more affordable than
traditional classroom education. There is also often a wide range of
payment options that let you pay in installments or per class. This
allows for financially disadvantaged students to upskill themselves to
open avenues for better career opportunities. Money can also be saved
from the commute and class materials which are often available for
free. In other words, the monetary investment is less, but the results
can be comparable to other options.
\fi

Online education allows students to determine their own learning speed and has the extra benefit of allowing them to create a timetable that matches everyone's schedule. As a result, when students learn online, finding a good work-study balance is easy. Students in online classes who do not have access to a live teacher can still benefit from comments thanks to an aiding machine learning-based Static Code Analyser that provides individualised feedback.


