% Chapter 3

\chapter{BACKGROUND AND MOTIVATION} % Write in your own chapter title
%

\section{INTRODUCTION} % Write in your own chapter title
%\label{fig:INTRODUCTION}
%\lhead{CHAPTER 1. \emph{INTRODUCTION}} % Write in your own chapter title to set the page header
Static Code Analysis (also known as Source Code Analysis) is
usually performed as part of a Code Review (also known as
white-box testing). Static Code Analysis in general refers to
the use of Static Code Analysis tools which try to highlight
potential vulnerabilities within `static' (non-running)
source code. This process provides better understanding of
structure of the code and can help ensure that the code
adheres to industry standards.

Software development teams and quality assurance teams employ
static analysis to discover possible vulnerabilities.  While
validating the code, the software will search all of the code
in a project for vulnerabilities. Static analysis is often
effective in detecting coding flaws such as programming
errors, coding standards violations, and security flaws.

Static code analysis has several advantages including
improving code quality by evaluating all of the code in an
application. When compared to manual code review, it allows
for faster use of automated tools. Static testing, when
combined with traditional testing methods, allows for greater
debugging depth. Human error is less likely with automated
tools. It will improve online or application security by
raising the possibility of identifying vulnerabilities in the
code.

We propose to use Machine Learning to perform Static Code
Analysis (SCA) on student coding assignments to evaluate the
code and give valuable feedback to improve the quality of
code. We propose to accomplish this by training a new
pipeline on our manually annotated student assignment data
set. In our study, the pipeline will produce a design value
score ranging from 1 to 10, which will be utilised to give
personalised feedback explaining the logic behind the
predicted design value score.

As a result, because feedback is a crucial part of effective
learning, the review or feedback provided helps students
improve the quality of their code. According to research, in
the context of student assignments, feedback is more strongly
and consistently connected with achievement than any other
teaching practice. This link exists regardless of grade,
financial status, race, or educational setting. Students'
self-esteem, self-awareness, and drive to learn can all
benefit from feedback. This method of providing feedback to
students has been proposed as a strategy to improve learning
and evaluation performance.

\section{MOTIVATION} % Write in your own chapter title
Feedback is an essential component of scaffolding for learning as
feedback assists students in achieving their learning goals and to
improve their self-regulation capability. The findings show that
participants were more engaged during the assessment when the feedback
was valuable and the explanations were clear and helpful.

Especially in recent situtaions of online learning due the pandemic,
feedback becomes even more critical since instructors and students are
separated geographically. In this case, feedback enables the
instructor to tailor learning content to the needs of the
pupils. Giving feedback, on the other hand, is a difficult chore for
instructors, especially in large groups. As a result, numerous
automatic feedback systems have been proposed in order to lessen the
instructor's labor.

Grading assignments manually is a very tedious and also a error-prone
task. Since grading is typically done for a large number of
student submissions, there is a increased probability of errors. The
use of artificial intelligence can be useful to address these issues
by automating the grading process. We can assist teachers in the
correction and enable students to receive immediate feedback which
will help the students to improve their program solutions before their
final submission.

\iffalse
Online education in general is less expensive and more affordable than
traditional classroom education. There is also often a wide range of
payment options that let you pay in installments or per class. This
allows for financially disadvantaged students to upskill themselves to
open avenues for better career opportunities. Money can also be saved
from the commute and class materials which are often available for
free. In other words, the monetary investment is less, but the results
can be comparable to other options.
\fi

Online education enables the student to set their own learning pace,
and there's the added flexibility of setting a schedule that fits
everyone's agenda. As a result, finding a solid work-study balance is
easier when the students study online. Students of online courses that
lack a real-time tutor available can also reap the benefits of
feedback with an assisting machine learning based Static Code Analyser
that gives personalized feedback to students.


