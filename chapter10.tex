\chapter{CONCLUSION}

We proposed a pipeline for grading a student code submission out of 10
which will also provide appropriate feedback to improve the code. The student programs were represented using 17 features which were retrieved from the submitted codes and from the AST representation of the codes. Feature set for each coding
assignment was optimised via feature selection. The initial dataset
created by us was used to train three regressor models such as Linear SVR, MLP regressor and Random Forest
Regressor. The Random Forest regressor was shown to be the most
effective for this problem statement. The Random Forest Regressor model
has a MAE of 1.7 and RMSE of 2.3.

Furthermore, our technique delivers personalised feedback for a program by comparing its feature vector with the average feature vector of the "excellent" programs submitted by students.Whenever a student submits a
code, he will be given feedback instantly to help him improve his
code.  Students can also view the best code submission after the
assignment deadline. The Streamlit framework was used to deploy this
pipeline as a web app.
